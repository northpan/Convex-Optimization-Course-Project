\documentclass[11pt, oneside]{article}   	% use "amsart" instead of "article" for AMSLaTeX format
\usepackage{geometry}                		% See geometry.pdf to learn the layout options. There are lots.
\geometry{letterpaper}                   		% ... or a4paper or a5paper or ... 
%\geometry{landscape}                		% Activate for rotated page geometry
%\usepackage[parfill]{parskip}    		% Activate to begin paragraphs with an empty line rather than an indent
\usepackage{graphicx}				% Use pdf, png, jpg, or eps§ with pdflatex; use eps in DVI mode
								% TeX will automatically convert eps --> pdf in pdflatex	
\usepackage{amssymb}
\usepackage{amsmath}
\usepackage[linesnumbered,boxed,ruled,commentsnumbered]{algorithm2e}
\usepackage{listings}
%SetFonts

%SetFonts


\title{Convex Optimization Homework 5, Assignment 3}
\author{Dinghuai Zhang, School of Mathematical Sciences, 1600013525}
\date{}							% Activate to display a given date or no date

\begin{document}
\maketitle
\section{Problem Settings}
\begin{equation}
\min_x \frac{1}{2}\Vert Ax-b\Vert^2_2+\mu\Vert x\Vert_1
\end{equation}
It's a standard LASSO problem without constraints.
\subsection{Data Building}
\lstset{
 frame=single, 
breaklines=true,
language=MATLAB,
 }
\begin{lstlisting}
n = 1024;
m = 512;
A = randn(m,n);
u = sprandn(n,1,0.1);
b = A*u;
mu = 1e-3;
x0 = rand(n,1);
\end{lstlisting}

\section{Solve the dual problem using Augmented Lagrangian method}
The dual problem of lasso is:
\begin{align}
\min & -b^Ty+\frac{1}{2}||y||^2_2 \\
s.t. & ||A^Ty||_{\infty}\le \mu
\end{align}
which can be formulated as:
\begin{align}
\min &\ f(y) = -b^Ty+\frac{1}{2}||y||^2_2 + \textrm{I}_{\{||z||_{\infty}\le\mu\}}\\
s.t. &  \ z = A^Ty
\end{align}

Impose augmented lagrangian method to this dual problem we solve 
\begin{align}
\left( y ^ { + } , z^ { + } \right) =& \ \underset { y , z } { \operatorname { argmin } } \  -b^Ty+\frac{1}{2}||y||^2_2 +x^T( A^Ty-z ) + \frac { \beta} { 2 } || A^Ty-z || ^ { 2 } _2\\
=& \ \underset { y , z } { \operatorname { argmin } } \  -b^Ty+\frac{1}{2}||y||^2_2 + \frac { \beta} { 2 } || (A^Ty+\frac{x}{\beta})-z || ^ { 2 } _2\\
\text { s.t. } & \ ||z||_{\infty}\le\mu
\end{align}
with:\\
\begin{algorithm}[H]
  \SetAlgoNoLine
  \caption{Augmented Lagrangian method for the dual problem} 
  \KwIn{$\gamma, \beta$, $y$, $x$, $N$, $z$} 
  \KwOut{solution $f(y)$} 
  initializing $k = 1$\\
  \While{ $k<N$} 
  {  
  $y\gets \underset { y } { \operatorname { argmin } } \{-b^Ty+\frac{1}{2}||y||^2_2 +   \frac{\beta}{2}|| (A^Ty+\frac{x}{\beta}) - \phi(A^Ty+\frac{x}{\beta})||^2_2\}$\\
  $z \gets \phi(A^Ty+\frac{x}{\beta})$\\
  $ x\gets x - \gamma (A^Ty-z)$\\
   $k\gets k+1$
  }
  return $f(y)$
\end{algorithm}
where $(\phi ( u ))_i = \left\{ \begin{array} { l l } { \mu } & { u _ { i } \geqslant \mu } \\ { u _ { i } } & { \text { otherwise } } \\ { - \mu } & { u _ { i } \leqslant - \mu } \end{array} \right.$ 
is the projection on a $\infty$-norm ball.\\
(It's easy to see that $\psi(u)=u-\phi(u)$ is soft-threshholding.)

For the $y$-sub problem, we can simply do gradient descent (or newton method), the gradient of $y$ is grad $= y - b + \beta A((A^Ty+\frac{x}{\beta}) - \phi(A^Ty+\frac{x}{\beta}))$, so we simply do $y\gets y - 0.1*\textrm{grad}$.
 
The code is showed as follows:
\lstset{
 frame=single, 
breaklines=true,
language=MATLAB,
 }
\begin{lstlisting}
function [x, out] = l1_auglagrange_dual(x0, A, b, mu, opts)
[m, n] = size(A);
pseudomu = 10000*mu;
beta = opts(1);
gamma = opts(2);
maxIter1 = opts(3);
maxIter2 = opts(4);
tol = 1e-6;

x = zeros(n, 1);
y = zeros(m, 1);

%continuation trick
while pseudomu >= mu     
    iter1 = 1;
    while iter1 < maxIter1
        % y sub-problem: gradient descent       
        iter2 = 1;
        while iter2 < maxIter2
            temp = softThreshold(A'*y+x/beta, pseudomu);
            grad = y - b + beta*A*temp;
            y = y - 0.1 * grad;

            iter2 = iter2 + 1;
         end
        
        temp=x/beta+A'*y;
        z = phi(temp, pseudomu); 
        x = x + gamma*(A'*y-z);
                
        iter1 = iter1 + 1;
    end
pseudomu = pseudomu / 10;
end
out = 0.5 * sum_square(A*x-b) + mu * norm(x, 1);
end

function [x] = softThreshold(x, mu)
    x = sign(x) .* max(abs(x) - mu, 0);
end

function ret = phi(input, mu)
input = min(input, mu);
input = max(input, -mu);
ret = input;
end
\end{lstlisting}

\section{Solve the dual problem using Alternating direction method of multipliers}
The problem is still
\begin{align}
\min &\ f(y) = -b^Ty+\frac{1}{2}||y||^2_2 + \textrm{I}_{\{||z||_{\infty}\le\mu\}}\\
s.t. &  \ z = A^Ty
\end{align}
In every iteration:
\begin{align}
y = &\ \underset { y } { \operatorname { argmin } } -b^Ty+\frac{1}{2}||y||^2_2+\textrm{I}_{\{||z||_{\infty}\le\mu\}} + \frac{\beta}{2}|| (A^Ty+\frac{x}{\beta}) - z||^2_2\\
= & \ (I + \beta AA^T)^{-1}(\beta Az+b-Ax)\\
z =&\  \underset { z} { \operatorname { argmin } } -b^Ty+\frac{1}{2}||y||^2_2 +\textrm{I}_{\{||z||_{\infty}\le\mu\}}+ \frac{\beta}{2}|| (A^Ty+\frac{x}{\beta}) - z||^2_2\\
= &\ \phi(A^Ty+\frac{x}{\beta})\\
x = & \ x + \gamma(A^Ty-z)
\end{align}

\begin{algorithm}[H]
  \SetAlgoNoLine
  \caption{Augmenteddirection method of multipliers for the dual problem} 
  \KwIn{$\gamma, \beta$, $y$, $x$, $N$, $z$} 
  \KwOut{solution $f(y)$} 
  initializing $k = 1$\\
  \While{ $k<N$} 
  {  
  $y\gets  (I + \beta AA^T)^{-1}(\beta Az+b-Ax)$\\
  
  $z \gets  \phi(A^Ty+\frac{x}{\beta})$\\
  $ x\gets x + \gamma (A^Ty-z)$\\
   $k\gets k+1$
  }
  return $f(y)$
\end{algorithm}

The code is showed as follows:
\lstset{
 frame=single, 
breaklines=true,
language=MATLAB,
 }
\begin{lstlisting}
function [x, out] = l1_admm_dual(x0, A, b, mu, opts)
[m, n] = size(A);
pseudomu = 10000*mu;
beta = opts(1);
gamma = opts(2);
maxIter = opts(3);
inver = inv(beta*(A*A') + eye(m));
x = zeros(n, 1);
z = zeros(n, 1);


while pseudomu >= mu
     
    iter = 0;
    
    while iter < maxIter                
        y = inver * (beta*A*z + b - A*x);
        z = phi(A'*y + x/beta, pseudomu);
        x = x + gamma*(A'*y-z); 

        iter = iter + 1;
    end
    pseudomu = pseudomu / 10;
end
out = 0.5 * sum_square(A*x-b) + mu * norm(x, 1);
end

function ret = phi(input, mu)
input = min(input, mu);
input = max(input, -mu);
ret = input;
end
\end{lstlisting}

\section{Solve the dual problem using Alternating direction method of multipliers with linearization}
Reformulate lasso as:
\begin{align}
\min &\ \frac{1}{2}\Vert Ax-b\Vert^2_2+\mu\Vert z\Vert_1 \\
s.t. & \ x - z =0
\end{align}
The augmented lagrangian is :
\begin{equation}
\frac{1}{2}\Vert Ax-b\Vert^2_2+\mu\Vert z\Vert_1  + \frac{\beta}{2}||x-z+u||^2_2 
\end{equation}
So the admm iteration step is:
\begin{align}
x \gets & (A^TA+\beta I)^{-1}(A^Tb+\beta(z - u))\\
z \gets & \textrm{soft-thresholding}(x+u, \frac{\mu}{\beta})\\
u \gets & u + \gamma(x -z )
\end{align}
If we adopt linearization, the update for $x$ is changed to:
\begin{align}
x \gets &  \  { x } - c \left( \nabla f \left( { x }  \right) + \beta   \left(  x-z+y\right) \right)\\
 = &\ x - c\left((A^TA+\beta I)x+\beta(u-z)-A^Tb\right)
\end{align}

The code is showed as follows:
\lstset{
 frame=single, 
breaklines=true,
language=MATLAB,
 }
\begin{lstlisting}
function [x, out] = l1_admm_primal_linear(x0, A, b, mu, opts)
[m, n] = size(A);
pseudomu = 10000*mu;
beta = opts(1);
gamma = opts(2);
maxIter = opts(3);
c = opts(4);
tol = 1e-6; 

x = x0;
u = randn(n, 1);
Q = inv(A'*A+beta*eye(n));
Atb = A' * b;

thresh = @(x,th) sign(x).* max(abs(x) - th,0);

while pseudomu >= mu
    for iter = 1:maxIter
        z = thresh(x + u, mu/beta);
        x = x - c*(AtA*x + beta*x + beta*(u-z) - Atb);
        u = u + gamma * (x - z); 
    end
    pseudomu = pseudomu / 10;  
end
out = 0.5 * sum_square(A*x-b) + mu * norm(x, 1);
end

\end{lstlisting}

\section{Numerical results and interpretations}
We can see the cpu time, the optimal value and error with cvx mosek in the tabluation:

\begin{table}[h]
  \centering
  \begin{tabular}{c|ccc}
  \hline
  Method&cpu time/s&error with cvx-mosek&optimal value\\
  \hline
  cvx-call-mosek&0.98&0.00e+00&7.2844266239e-02\\
  auglagrange dual&0.19&2.97e-06&7.2844322774e-02\\
  admm dual&0.17&2.75e-06&7.2844268531e-02\\
  admm primal linear&0.65&3.04e-06&7.2844295011e-02\\
  \hline
  \end{tabular}
\end{table}

Alternative methods can achieve good results compared to cvx. However when implemented with linearization, the efficiency will be a little worse because of the approximation.

\end{document}  