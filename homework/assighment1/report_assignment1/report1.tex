\documentclass[11pt, oneside]{article}   	% use "amsart" instead of "article" for AMSLaTeX format
\usepackage{geometry}                		% See geometry.pdf to learn the layout options. There are lots.
\geometry{letterpaper}                   		% ... or a4paper or a5paper or ... 
%\geometry{landscape}                		% Activate for rotated page geometry
%\usepackage[parfill]{parskip}    		% Activate to begin paragraphs with an empty line rather than an indent
\usepackage{graphicx}				% Use pdf, png, jpg, or eps§ with pdflatex; use eps in DVI mode
								% TeX will automatically convert eps --> pdf in pdflatex	
\usepackage{amssymb}
\usepackage{amsmath}

\usepackage{listings}
%SetFonts

%SetFonts


\title{Convex Optimization Homework 5}
\author{Dinghuai Zhang, School of Mathematical Sciences, 1600013525}
\date{}							% Activate to display a given date or no date

\begin{document}
\maketitle
\section{Problem Settings}
\begin{equation}
\min_x \frac{1}{2}\Vert Ax-b\Vert^2_2+\mu\Vert x\Vert_1
\end{equation}
It's a standard LASSO problem without constraints.
\subsection{Data Building}
\lstset{
 frame=single, 
breaklines=true,
language=MATLAB,
 }
\begin{lstlisting}
n = 1024;
m = 512;
A = randn(m,n);
u = sprandn(n,1,0.1);
b = A*u;
mu = 1e-3;
x0 = rand(n,1);
\end{lstlisting}

\section{Solve the problem using CVX calling Gurobi and Mosek}

\subsection{Solving the problem using Mosek in CVX}
The code is showed as follows:
\lstset{
 frame=single, 
breaklines=true,
language=MATLAB,
 }
\begin{lstlisting}
function [ x1,out1 ] = l1_cvx_mosek(x0, A, b, mu, opts1)
[m,n] = size(A);
cvx_solver mosek
cvx_begin
    variable x(n)
    minimize (0.5*square_pos(norm(A*x-b))+mu*norm(x,1))
cvx_end
x1 = x;
out1 = cvx_optval;
end
\end{lstlisting}

The result is 
\lstset{
 frame=single, 
breaklines=true,
language=MATLAB,
 }
\begin{lstlisting}
Optimizer summary
  Optimizer                 -                        time: 0.59    
    Interior-point          - iterations : 8         time: 0.54   
Optimal value (cvx_optval): +0.0746955  
\end{lstlisting}

\subsection{Solving the problem using Gurobi in CVX}
The code is showed as follows:
\lstset{
 frame=single, 
breaklines=true,
language=MATLAB,
 }
\begin{lstlisting}
function [x, cvx_optval]=l1_cvx_gurobi(x0, A, b, mu, opts2)
[n, ~] = size(x0);
cvx_solver gurobi
cvx_begin 
    variable x(n)
    minimize (mu*norm(x,1) + 0.5*norm(A*x-b))    
cvx_end
end
\end{lstlisting}

The result is 
\lstset{
 frame=single, 
breaklines=true,
language=MATLAB,
 }
\begin{lstlisting}
Barrier solved model in 13 iterations and 1.04 seconds
Optimal objective 7.46954736e-02 
\end{lstlisting}

\section{Solve the problem by directly calling Gurobi and Mosek}
The original problem can be reformulated as:
    \begin{align}
    \min &\frac{1}{2}\Vert Ax-b\Vert^2_2+\mu1^Tt \\
     s.t.&  x\preceq t\\
    \quad -&t\preceq x
    \end{align}
which is (when $b = Au$):
    \begin{align}
    \min &\frac{1}{2}\Vert Ay\Vert^2_2+\mu1^Tt \\
     s.t.&  y+u\preceq t\\
    \quad -&t\preceq y+u
    \end{align}
In this way we can call mosek and gurobi directly.

\subsection{Solve the problem by directly calling Mosek}
The code is showed as follows:
\lstset{
 frame=single, 
breaklines=true,
language=MATLAB,
 }
\begin{lstlisting}
function [x, cvx_optval]=l1_mosek(x0, A, b, mu, opts3)
[~,n] = size(A);
q = [0.5*(A'*A), zeros(n,n);zeros(n,n),zeros(n,n)];
c = [zeros(n,1);mu*ones(n,1)];
a = [eye(n),eye(n);-eye(n),eye(n)];
v = A\b;
blc = [-v;v];
res = mskqpopt(q,c,a,blc,[],[],[],[],'minimize');
x = res.sol.itr.xx(1:1024)+v;
cvx_optval = res.sol.itr.pobjval;
end
\end{lstlisting}

The result is 
\lstset{
 frame=single, 
breaklines=true,
language=MATLAB,
 }
\begin{lstlisting}
 Optimizer                 -                        time: 1.31    
    Interior-point          - iterations : 10        time: 1.25    
 Optimal value (cvx_optval): +0.0746955
\end{lstlisting}

\subsection{Solve the problem by directly calling Gurobi}
The code is showed as follows:
\lstset{
 frame=single, 
breaklines=true,
language=MATLAB,
 }
\begin{lstlisting}
function [x, cvx_optval]=l1_gurobi(x0, A, b, mu, opts4)
[~,n] = size(A);
At = [A -A];
model.Q = 0.5*sparse(At'*At);
c = mu*ones(2*n,1)- (b'*At)';
model.obj = c;
P = eye(2*n);
model.A = sparse(P);
l1 = zeros(2*n,1);
model.rhs = full(l1);
model.sense = '>';
params.method = 2;
res = gurobi(model, params);
x = res.x(1:n,1)-res.x(n+1:2*n,1);
cvx_optval = 0.5 * sum_square(A * x - b) + mu * norm(x, 1);
end
\end{lstlisting}

The result is 
\lstset{
 frame=single, 
breaklines=true,
language=MATLAB,
 }
\begin{lstlisting}
Barrier solved model in 9 iterations and 1.52 seconds
Optimal objective 7.4696220351e-02
\end{lstlisting}

\section{Solve the problem with Projection Gradient Algorithm}
The core iteration step of projection gradient algorithm is:
\begin{equation}
x ^ { + } = P _ { C } ( x - t \nabla g ( x ) )
\end{equation}
We separate variable $x$ into two parts of $x_{1}, x_{2}$ \textit{s.t.} $x_1\ge0, x_{2}\ge0$ and define $C$ as $\mathbb{R^{+}}_{2n}$.
In the implementation we start from a large $\mu$ to ensure its convergence. The code is showed as follows:
\lstset{
 frame=single, 
breaklines=true,
language=MATLAB,
 }
\begin{lstlisting}
x0 = [max(x0,0); max(-x0,0)];
[~,n] = size(A);
pseodumu = mu * 1e5;
tol = 1e-4;
alpha_l = 1e-6; 
alpha_u = 1;    
alpha = 1e-4;     
iter = 200;
Atb = A' * b;
AtA = A' * A;
while pseodumu >= mu
    cur = pseodumu * ones(2*n,1) - [Atb; -Atb];
    AtAx0 = AtA* (x0(1:n)-x0((n+1):2*n));
    g0 = [AtAx0;-AtAx0] + cur;
    x = x0;
    g = g0;
    k = 1;
    while k < iter
        x0 = x;
        x = max(x - alpha*g, 0);    
        g0 = g;
        AAz = AtA* (x(1:n)-x((n+1):2*n));
        g = [AAz;-AAz] + cur;
        y = g - g0;
        s = x - x0;
        BB = (s'*s) / (s'*y); 
        if s'*y <=0
            alpha = alpha_u;
        else
            alpha = max(alpha_l, min(BB, alpha_u));
        end
        k = k + 1;
        if  norm(max(x-g, 0) - x) < tol 
            break
        end
    end
    pseodumu = pseodumu / 10; 
end
x =  x(1:n)-x((n+1):2*n);
cvx_optval = 0.5*sum_square(A*x - b) + mu*norm(x,1);
end
\end{lstlisting}

The result is \textbf{better} and \textbf{faster} than mosek's 7.4695473580e-02 in 1.16s:
\lstset{
 frame=single, 
breaklines=true,
language=MATLAB,
 }
\begin{lstlisting}
sub-gradient: cpu:  0.46, cvx_optval: 7.4695410841e-02
\end{lstlisting}


\section{Solve the problem with Subgradient Algorithm}
The sub-gradient algorithm is showed as follows:
\begin{equation}
x ^ { ( k ) } = x ^ { ( k - 1 ) } - t _ { k } g ^ { ( k - 1 ) }
\end{equation}
where $g^{(k-1)}$ is the sub gradient of object function when $x = x^{(k-1)}$.
In the implementation we start from a large $\mu$ to ensure its convergence as above. What's more, we use continuation method for step length $\alpha$. The code is showed as follows:
\lstset{
 frame=single, 
breaklines=true,
language=MATLAB,
 }
\begin{lstlisting}
function [x, cvx_optval]=l1_Subgradient(x0, A, b, mu, opts5)
pseudomu = 100; 
iter = 500; 
tol = 1e-7;
x = x0;

while pseudomu >= mu
    k = 1;
    alpha = 3e-4;
    while k < iter
        x0 = x;
        g = A' * (A * x - b)  + pseudomu * sign(x);
        if mod(k,50)==0
            alpha = alpha * 0.9;
        end
        x = x - alpha * g;
        if norm(x0-x) < tol
            break;
        end
        k = k + 1;
    end
    pseudomu = pseudomu / 10;
end
cvx_optval = 0.5*sum_square(A*x - b) + mu * norm(x,1);
end
\end{lstlisting}

The result is \textbf{better} and \textbf{faster} than mosek's 7.4695473580e-02 in 1.16s:
\lstset{
 frame=single, 
breaklines=true,
language=MATLAB,
 }
\begin{lstlisting}
projectgradient: cpu:  0.11, cvx_optval: 7.4695365639e-02
\end{lstlisting}

\end{document}  